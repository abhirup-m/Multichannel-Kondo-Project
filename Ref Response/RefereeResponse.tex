\documentclass{article}
\usepackage{amsmath,amssymb,braket}
\usepackage{color}
\usepackage{xcolor}
\usepackage[legalpaper, portrait]{geometry}
\DeclareUnicodeCharacter{2212}{-}
\DeclareUnicodeCharacter{03B8}{+}
\newcommand{\shat}{\hat{\mathbf{s}}}
\newcommand{\pin}{\par\noindent}
\newcommand{\response}[1]{{\color{blue}\subsection*{Response:}{#1}}}
\renewcommand{\thesubsection}{Point \arabic{subsection}.}
\newcommand{\point}[1]{\subsection{}{#1}}

\author{Siddhartha Patra, Abhirup Mukherjee, Anirban Mukherjee, N. S. Vidhyadhiraja,\\ A. Taraphder, S Lal}

\title{Referee response to BS14324 Patra}
\begin{document}
\maketitle

We thank the three referees for comments, questions and suggestions. These have helped substantially in improving the overall presentation of the manuscript. Below, we present our responses (in blue) to the points raised by the three referees in turn.

\section{First referee}

\point{In general, the paper is comprehensive and well-organized, though I would encourage the authors to go over the manuscript in the next revision and correct several typos and inconsistencies in notation, few of which I list below.}
\response{We thank the referee for the comment. We have corrected the typographic errors and notational inconsistencies in the manuscript.} 

\point{
Examples of minor issues:
\response{ We thank the referee for pointing out these typographic errors.} 
\begin{itemize}
	\item Eq.(3), are both the commutator and anticommutator of the $\eta$
operators equal to 1, or just the anticommutator?\\
{\color{blue}{\bf Response:} We thank the referee for pointing out these typographic errors. Indeed, only the anticommutator is unity, while the commutator is actually equal to $2\hat n_j - 1$~\cite{anirbanurg1}. We have updated the manuscript with this correction.}
	\item In several equations, the sum is over $\alpha$ and $\beta,$ but the operators have $\alpha$ and $\alpha^\prime$ indices.\\
{\color{blue}{\bf Response:} We have gone through the entire manuscript and made sure that the notation is now consistent.}
	\item Some notations, like $\theta_M$ above Eq. 16 is introduced and never
	used.\\
{\color{blue}{\bf Response:} We have removed redundant notation here and elsewhere in the manuscript, wherever we could find.}
\end{itemize}
}

\point{
Can this technique of using the star graph together with effective Hamiltonian for low energy excitations be also used to obtain dynamical properties such as conductance? As this is often one of the most important experimental probes used to study Kondo problems, I would be interested to learn if URG can provide additional insight.}

\response{ We are grateful to the referee for this question. Indeed, we have used the URG method in the past to compute certain dynamic quantities for the single-channel Kondo model in an earlier work~\cite{kondo_urg}, e.g., the impurity Greens function and spectral functions at both zero and finite temperatures. This was done by working with the fixed point Hamiltonian obtained from the URG flow (and without taking the zero bandwidth limit). The spectral function revealed the expected Abrikosov-Suhl resonance at zero frequency, and satisfied the Friedel sum rule. We believe that computations can also be performed using the complete fixed point Hamiltonian (i.e., not just that of the star graph) in order to obtain other Greens functions that yield dynamical quantities like the conductance or the resistance. We propose to develop this in a future work.}

\point{
In several places, the authors mention performing numerical calculations to verify the obtained results (for example paragraph above Eq 19 about impurity susceptibility). What method was used to perform the calculations? Is it just evaluation of for example Eq 15 for the Heisenberg Hamiltonian of the star graph or is the numerical calculation itself performing the renormalization treatment?}

\response{ In the example cited in the question, we used numerical calculations in the sense of solving the star graph Hamiltonian using exact diagonalisation and then using Eq.(19) to obtain the susceptibility. In the Supplementary Material~\cite{SM}, in order to obtain the contribution to the susceptibility coming from the non-Fermi liquid, we numerically diagonalise the Hamiltonian formed of the star graph and the single-particle excitations into the conduction bath. We have now clarified the methods employed below eq.(18) as well as below Fig.5 in the main manuscript and supplementary materials (above eq.(40)).}

\point{
Expanding on the previous point, can unitary renormalization group approach give rise to some new numerical technique akin to numerical renormalization group?}

\response{We thank the referee for this excellent question. In the present work (as well as our earlier work on the single channel Kondo effect~\cite{kondo_urg}), we have emphasised obtaining various measurable thermodynamic and dynamical quantities of the quantum impurity model from the numerical diagonalisation of the fixed point effective Hamiltonian of the URG flow. In Ref.\cite{kondo_urg}, we found that this gave reasonably good qualitative agreement with the results obtained from the NRG method (in terms of the asymptotic values and the nature of the crossover). However, it is also possible to compute various quantities at several points along the URG flow, in the spirit of the NRG procedure. Implementing this would, we believe, take us closer to a technique akin to NRG. 
\par\noindent
Importantly, given that the URG can be (and has already been) applied to a variety of models across dimensions, this would help overcome the major limitation of NRG in being applicable only to quantum impurity models. It will be interesting to see the trade-off between an increase in quantitative accuracy versus the complexity of the numerical implementation. This challenge we leave to the future. However, it is worth pointing out that we have already adopted the philosophy mentioned above in implementing a momentum-space entanglement renormalisation group (MERG) scheme based on the URG method. Indeed, we have applied the MERG to the study of several problems of strongly correlated electrons (e.g., the 2D Hubbard model on the square lattice at and away from $1/2$-filling~\cite{mukherjeeMERG2022}, the single channel Kondo problem~\cite{kondo_urg}, the 1D Hubbard model~\cite{1dhubjhep} and the 2D reduced BCS model~\cite{siddharthacpi}).}

\section{Second referee}

\point{
My main comments concern Section II-A, in which the authors define the RG procedure. Since the results of the other Sections follow from the results of this Section, I would suggest the authors expand it. Also, some of the results that come later using this technique are familiar to the experts in the field of heavy fermions, but the RG procedure may not be.}

\response{We thank the referee for raising this point. Given the present length of the work, it will be difficult for us to add substantial material devoted to explaining various features of the URG method in the main manuscript. Instead, we have improved Sec.4 of the Supplementary Materials (which contains the detailed derivation of the URG flow equations) to cover this. In addition, we have addressed within the main manuscript the specific questions that the referee has brought up in the two points immediately following this one (and which are related to the URG method). Finally, we have also pointed the readers towards a complete derivation of the URG formalism in Refs.~\cite{anirbanurg1,anirbanurg2}. We earnestly request the referee to kindly bear with us on this.}


\point{
It would benefit the reader of this article to get some intuition about the generators, Eqs. (3) and (4). Is there a simple way to understand this choice?}

\response{The unitary operator \(U\) mentioned in Eq.(3) can be thought of as a special case of the most general form \(U = e^\mathcal{S}\) of a unitary operator defined by an anti-Hermitian generator \(\mathcal{S}\). As shown in detail in Ref.\cite{anirbanurg1}, for our case, the generator turns out to be \(\mathcal{S} = \frac{\pi}{4}\left( \eta^\dagger - \eta \right) \), such that the unitary operator is \(U = e^{\frac{\pi}{4}\left(\eta^\dagger - \eta\right) }\). If one then uses the anticommutation and commutation properties of \(\eta,\eta^\dagger\), the exponential can be expanded in a Taylor series and then resummed into the form mentioned in the manuscript.
\par\noindent
The fermionic operators \(\eta_{(j)},\eta^\dagger_{(j)}\) of Eq.(4) can be thought of as the many-particle analogue of the single-particle field operators \(c_j,c^\dagger_j\) that change the occupation number of the single-particle Fock space \(\ket{n_j}\). The presence of the operators in $\eta_{(j)}$ other than $c_j$ are required to ensure that the decoupling of the specific Fock state under consideration is carried out through a unitary modification of the spectrum of the remnant degrees of freedom. Since the \(j^\text{th}\) degree of freedom is coupled with many others, a rotation of the basis of \(j\) also involves a rotation of the many-particle basis of the remaining degrees of freedom in the Hamiltonian \(H_{(j)}\); this is encoded within the \(\text{Tr}\left(H_{(j)} c_j\right) \) term. The Greens function in front of $c_j$ accounts for the dynamical cost of this rotation in terms of an operator for quantum fluctuations (\(\hat \omega\)), as well as the kinetic and self-energies \(\text{Tr}\left( H_{(j)}\hat n_j \right) \).
\par\noindent
We have updated the manuscript with these explanations in Sec.IIA, below eqs.(3), (4) and (5). We are thankful to the referee for these questions.
}

\point{Other RG proposals come from applying a set of successive unitary transformations (such as the Wegner RG, as discussed in detail in ``The Flow Equation Approach to Many-Particle Systems" by Stefan Kehrein). How does this approach differ from those?}

\response{
There are atleast two specific distinctions between the continuous unitary transformations (CUT) RG as envisaged by Wegner and the URG of the present work.
\begin{itemize}
	\item While most RG methods that apply successive unitary transformations rely on a gradual decrease in the off-diagonal content of the Hamiltonian, the URG proceeds by decoupling each discrete Fock space node at a time. In other words, the URG achieves a block-diagonalisation of the Hamiltonian at each RG step, while the CUT RG involves a gradual decay of the off-diagonal terms of the Hamiltonian in order to make it more band diagonal.
	\item The CUT RG involves choosing a truncation scheme that allows closing the RG equations. The URG, on the other hand, is able to resum the coupling expansion series through the denominator structure and the quantum fluctuation operator.  
\end{itemize}
We have improved the manuscript by adding these comparisons. Comparisons of the URG with other methods like the functional RG or the spectrum bifurcation RG can also be found in Refs.~\cite{anirbanmott1,anirbanurg1}.
}

\point{Also, in Section II, the authors denote by $\ket{j}$ the highest energy electronic state. In this problem, assuming a quadratic dispersion to the electrons in 3 dimensions, the Fermi surface would generally be a sphere. Would $\ket{j}$ be the set of states with $k=k_{F}$?}

\response{We thank the referee for the question. At a particular RG step, the symbol \(\ket{j}\) is used to label the collection of the most energetic single particle Fock states of the bath in momentum-space that can be accessed from the ground state by virtual quantum fluctuations (and which are resolved by the decoupling procedure). In more conventional language, this collection of states lies at the running cutoff \(\Lambda\) (bandwidth) of the field theory. The bandwidth shrinks as the RG proceeds, and \(\ket{j}\) is updated to states at lower energies until it hits the Fermi surface (assuming the RG has not reached a fixed point by then).
}

\point{
This block in the Introduction is a bit hard to understand and maybe could be rewritten: ``the ground state is susceptible to perturbations that introduce channel isotropy in the exchange couplings [14, 32, 37, 45, 52]. This makes the experimental realisation of such states quite challenging." Do the authors mean channel anisotropy instead of channel isotropy?}

\response{ Indeed, we meant ``anisotropy" instead of isotropy. We have thus rectified the mistake. We thank the referee for pointing this out.}

\point{ 
After Eq. (7), implicit summation over the spin indices $\alpha,\beta$ should be $\alpha,\alpha^\prime$.}

\response{We thank the referee for pointing this out, and have rectified the mistake in the manuscript.}

\point{ 
In Eq. (8) and the text after Eq. (8), why are there two sums in the
k-expansion of S, over $k$ and $k^\prime$, as opposed to only one sum over k? Also, there is an issue with $\alpha^\prime$ and $\beta$ variables (similar to point 2.)}

\response{We have rectified the issue with the spin label \(\alpha^\prime\) label. With regards to the two momentum summations \(k,k^\prime\) in the expressions of \(\vec S\) in and below Eq.(8), both the summations are indeed required. This is because the expression for the spin operator in eq.(8) can be obtained by Fourier transforming the real-space local spin operator \(\sum_{\alpha\alpha^\prime}\vec \sigma_{\alpha\alpha^\prime}c^\dagger_{\alpha}c_{\alpha^\prime}\) formed by the conduction electrons \(c_{\alpha},c_{\alpha^\prime}\) at the origin. The Fourier transformation involves two $k$-space sums (one for each real-space operators).}
\point{
In Fig. 3, legend, one of the triangles should be a circle.}

\response{We have updated the plot with the correct legend. We are grateful to the referee for pointing this out.}

\point{
After Eq. (18), the authors attribute a 1/T contribution to the
susceptibility as a signature of the critical nature of the Hamiltonian. This 1/T (Curie)-behavior is present for the susceptibility of a single magnetic moment in a magnetic field. I don't see critical behavior in this. The log corrections to 1/T would be a signature of criticality.}

\response{We had (incorrectly) used the term critical to reflect the fact that the \(1/T\) divergence was a result of the remnant degeneracy of the system at the fixed point (reflecting an unscreened moment). In order to avoid any confusion on this point, we have removed the usage of the word ``critical" in this context from all places in the manuscript.}

\point{
Also, in Eq. (8), why does it vanish only for $K=1$, and not when $K=2S_{d}$, in general?}

\response{We believe the referee is referring to Eq.(18) here, and not Eq.(8). The expression in Eq.(18) was derived for the specific case of $S_d = 1/2$ (spin-half impurity), and hence it only vanishes at $K=2S_d=1$. while we do not have an analytic expression of this at present for the case $K=2S_{d}$, we have checked numerically that this is indeed the case. This is reported in Fig.(4).}

\point{
In Section IV A, the authors classify the terms as Fermi liquid or non-Fermi liquid. Is there a simple way to understand this classification? A possible way is by characterizing the effects of each term, but my question is if there is an immediate way of knowing if the term is FL or NFL.}

\response{ We thank the referee for seeking a clarification on this point. We used the term “non-Fermi liquid” to reflect the fact that, in the RG sense, the low energy $k−$space Hamiltonian was not purely of an $F_{kk^\prime} n_k n_{k^\prime}$ density-density type interaction term (which is characteristic of a Fermi liquid). Indeed, the off-diagonal terms in the effective Hamiltonian obtained in eq.(44) is shown to involve a dominant contribution corresponding to an effective Marginal Fermi liquid effective Hamiltonian (eq.(49), involving a $n_{k\sigma} n_{k^{\prime}\uparrow}(1-n_{k^{\prime}\downarrow})$ interaction term, and as shown in an earlier work by us on the 2D Hubbard model~\cite{anirbanmott1,anirbanmott2,anirbanurg1,anirbanurg2}).} 

\section{Third referee}

\point{
Perhaps the most interesting part is the “star graph” analysis which
are done rigorously as far as I can judge. However, I am not convinced
(especially not by the authors) that the strong coupling teaches
anything useful for the infrared physics. If there is a clear physics
lesson here, it is lost in the myriad of technicalities in the paper.
}

\response{We thank the referee for this question. Upon flowing from the weakly coupled theory to the fixed point under the renormalization group transformations, there is a reduction in the effective bandwidth of the problem (leading to reduced excitations into the conduction band) together with a growth in the effective Kondo coupling. This involves a growth in the ratio of the Kondo coupling to the bandwidth of the conduction band, and enables a study of the emergent star graph (obtained from a zero bandwidth approximation) as a ``skeletal" problem for the low-energy physics. Such an approach reveals, for instance, the ground state of the multichannel problem and the degeneracy. Moreover, it allows studying the effects of excitations (from a calculation in the spirit of Nozieres' calculation for the single channel Kondo problem), as demonstrated by the identification of its local non-Fermi liquid nature.
}

\point{
Derivation of the effective fixed
point Hamiltonian, cannot be trusted since there is no argument that
subsequent steps are going to make smaller changes (as opposed to
Wilson’s NRG).
}

\response{
We respectfully disagree with the referee on this point. As shown by us in Ref.\cite{anirbanurg1} in detail, the URG involves a resummation to all orders of renormalisations in various couplings. This is revealed by the non-perturbative structure of the URG equation. This ensures that the fixed point Hamiltonians obtained from URG indeed account for the effects of the UV on the IR, and are stable under renormalised perturbation theory (in the RG sense). This has been demonstrated by us in application of the URG to a variety of problems involving correlated electrons (included in the list of references in the main manuscript). Furthermore, in a recently published work of ours on the single channel Kondo problem~\cite{kondo_urg}, we have compared the results for several thermodynamic and dynamical measurables obtained from the URG against that obtained from methods such as NRG, and obtained very good agreement. 
}

\point{
It’s highly
technical and I believe it does not add anything to what we know and
furthermore, some steps and arguments are, in my opinion, not really
correct.
}

\response{
Again, we respectfully disagree with the referee. We believe that the following results are new and interesting additions to the knowledge of multichannel Kondo physics that are obtained from our work:
\begin{itemize}
	\item importance of the role of degeneracy and quantum-mechanical frustration in shaping the infrared physics of the MCK, specifically in the non-Fermi liquid behaviour, the thermodynamic properties, the orthogonality catastrophe, the breakdown of screening and in the strong-weak duality,
	\item derivation of a marginal Fermi liquid effective Hamiltonian for the non-Fermi liquid fixed point, hence revealing the microscopic source of the non-Fermi liquid behaviour in the overscreened regime,
	\item the presence of a spin Mott liquid in the effective interactions among the conduction electrons,
	\item demonstration of the orthogonality catastrophe in the ground state through entanglement measures.
\end{itemize}
The observation of the Mott liquid is particularly significant because it highlights a common theme that pervades much of fermionic quantum matter - the role of zero modes in determining the infrared physics. This can be seen, for instance, in the emergence of a Mott liquid from the 2D Hubbard model as seen from a URG analysis~\cite{anirbanmott1,anirbanmott2,mukherjeeMERG2022} and a Cooper pair insulator from the reduced 2D BCS Hamiltonian~\cite{siddharthacpi}. Importantly, we have also corroborated several of our analytic and numerical results with those that exist in the literature.
}

\section{Some additional changes in the manuscript}
We made the following additional changes in the manuscript, apart from those arising from the suggestions made by the referees (as mentioned above).
\begin{itemize}
	\item Updated an arxiv preprint reference to journal
	\item Modified acknowledgments
\end{itemize}

\bibliographystyle{unsrt}
\bibliography{../manuscript/mscript_mck_full.bib}
\end{document}
