\documentclass{revtex4-2}
\usepackage{braket,amsmath,amssymb,graphicx,float,hyperref}
\allowdisplaybreaks
\begin{document}
\title{Multi-channel Kondo model URG}
\author{Abhirup Mukherjee}
\date{\today}
\maketitle
\section{Introduction}
The multi-channel Kondo model is described by the Hamiltonian
\begin{equation}\begin{aligned}
	H = \sum_{k,\alpha,\gamma}\epsilon_{k}^\gamma \hat n^\gamma_{k\alpha} + J\sum_{kk^\prime,\gamma} \vec{S_d}\cdot\vec{s}_{\alpha\alpha^\prime}{c^\gamma_{k\alpha}}^\dagger c^\gamma_{k^\prime\alpha^\prime}~.
\end{aligned}\end{equation}
It is mostly identical to the single-channel Kondo model: \(k,k^\prime\) sum over the momentum states, \(\alpha,\alpha^\prime\) sum over the spin indices and \(\gamma\) sums over the various channels. \(\vec S_d, \vec s\) are the impurity and conduction bath spin vectors. The renormalization at step \(j\) is given by
\begin{equation}\begin{aligned}
	\Delta H_j = \left[c^\dagger T, \eta \right] = \left(c^\dagger T \frac{1}{\hat \omega - H_D}T^\dagger c - T^\dagger c \frac{1}{\hat \omega - H_D}c^\dagger T\right)
\end{aligned}\end{equation}
\begin{equation}\begin{aligned}
	c^\dagger T = J \sum_{k < \Lambda_j, \alpha}\vec{S_d}\cdot\vec{s}_{\beta \alpha}c^\dagger_{q\beta}c_{k\alpha}, &&&H_D = \epsilon_q \tau_{q\beta} + J S_d^z s_q^z
\end{aligned}\end{equation}
Usually we treat the \(\hat \omega\) as number(s) and study the renormalization in the couplings as functions of the quantum fluctuation scales. Each value of the fluctuation scale defines an eigendirection of \(\hat \omega\). We have then essentially traded off the complexity in the non-commuatativity of the diagonal and off-diagonal terms for all the directions in the manifold of \(\hat \omega\).

Here we will do something different. We will redefine the \(\hat \omega\) by pulling out the off-diagonal term from it: \(\hat \omega \to \hat \omega - H_X\), and then study the renormalization at various orders by expanding the denominator in powers of \(H_X\). Such a redefinition essentially amounts to a rotation of the eigendirections of \(\hat \omega\). This is done in order to extract some information out of \(\hat \omega\), specifically the dependence of the RG equations on the channel number \(K = \sum_\gamma\). This dependence is in principle present even if we do not do such a redefinition and expansion, in the various directions and values of \(\omega\), because those values encode the non-perturbative information regarding scattering at all loops. However, it is difficult to read off this information directly. This step of redefinition followed by expansion is being done with the sole aim of exposing such information. 

The expansion we are talking about is
\begin{equation}\begin{aligned}
	\eta = \frac{1}{\hat \omega - H_D}T^\dagger c = \frac{1}{\omega^\prime - H_D - H_X}T^\dagger c \simeq \frac{1}{\omega^\prime - H_D}T^\dagger c + \frac{1}{\omega^\prime - H_D}H_X \frac{1}{\omega^\prime - H_D} T^\dagger c
\end{aligned}\end{equation}
where \(H_X = J \sum_{k,k^\prime < \Lambda_j, \alpha,\alpha^\prime}\vec{S_d}\cdot\vec{s}_{\alpha \alpha^\prime}c^\dagger_{k\alpha}c_{k^\prime\alpha^\prime}\) is scattering between the entangled electrons.
With this change, the second and third order renormalizations will take the form
\begin{equation}\begin{aligned}
	\Delta H^{(2)}_j = c^\dagger T \frac{1}{\omega^\prime - H_D}T^\dagger c - T^\dagger c \frac{1}{\omega - H_D}c^\dagger T
\end{aligned}\end{equation}
\begin{equation}\begin{aligned}
	\Delta H^{(3)}_j = c^\dagger T \frac{1}{\omega^\prime - H_D} H_X \frac{1}{\omega^\prime - H_D} T^\dagger c - T^\dagger c \frac{1}{\omega - H_D} H_X \frac{1}{\omega - H_D} c^\dagger T
\end{aligned}\end{equation}

\section{Leading order renormalization}
\begin{equation}\begin{aligned}
	\Delta H^{(2)}_j = \underbrace{c^\dagger T \frac{1}{\omega^\prime - H_D}T^\dagger c}_\text{first term}~ -~ \underbrace{T^\dagger c \frac{1}{\omega - H_D}c^\dagger T}_\text{second term}
\end{aligned}\end{equation}
\subsection{Second term}
\begin{align}
	&T^\dagger c \frac{1}{\hat \omega - H_D}c^\dagger T \\
	&= J^2\sum_{q\beta k k^\prime \alpha \alpha^\prime} c^\dagger_{k^\prime\alpha^\prime} c_{q\beta} \vec{S_d}\cdot\vec{s}_{\alpha^\prime \beta } \frac{1}{\omega - \epsilon_q\tau_{q\beta} - J S_d^z s_q^z}c^\dagger_{q\beta} c_{k\alpha} \vec{S_d}\cdot\vec{s}_{ \beta\alpha}\\
	&= J^2\sum_{q\beta k k^\prime \alpha \alpha^\prime,a,b} c^\dagger_{k^\prime\alpha^\prime} c_{q\beta} S_d^a s^a_{\alpha^\prime \beta} \frac{1}{\omega - \frac{1}{2}D - J S_d^z s_q^z}c^\dagger_{q\beta} c_{k\alpha} S_d^b s^b_{\beta \alpha} & \left[\tau = \frac{1}{2}, \epsilon_q = D\right] \\
	&= J^2\sum_{q\beta k k^\prime \alpha \alpha^\prime,a,b} c^\dagger_{k^\prime\alpha^\prime} c_{q\beta} S_d^a s^a_{\alpha^\prime \beta} \frac{\omega - \frac{1}{2}D + J S_d^z s_q^z}{\left(\omega - \frac{1}{2}D\right)^2 - \frac{1}{16}J^2}c^\dagger_{q\beta} c_{k\alpha} S_d^b s^b_{\beta \alpha} \\
	&= J^2\sum_{q\beta k k^\prime \alpha \alpha^\prime,a,b} c^\dagger_{k^\prime\alpha^\prime} c_{k\alpha} S_d^a s^a_{\alpha^\prime \beta} \frac{\left(\omega - \frac{1}{2}D\right) \left(1 - \hat n_{q\beta}\right) +  J S_d^z c_{q\beta} s_q^z c^\dagger_{q\beta}}{\left(\omega - \frac{1}{2}D\right)^2 - \frac{1}{16}J^2} S_d^b s^b_{\beta \alpha} \\
	&= J^2\sum_{q\beta k k^\prime \alpha \alpha^\prime,a,b} c^\dagger_{k^\prime\alpha^\prime} c_{k\alpha} S_d^a s^a_{\alpha^\prime \beta} \frac{\left(\omega - \frac{1}{2}D\right) \left(1 - \hat n_{q\beta}\right) +  J S_d^z \left(s_q^z + \frac{\beta}{2}\right) \left(1- \hat n_{q\beta}\right)}{\left(\omega - \frac{1}{2}D\right)^2 - \frac{1}{16}J^2} S_d^b s^b_{\beta \alpha} & \left[\left[c_{q\beta}, s_q^z\right] = \frac{\beta}{2} c_{q\beta}\right] \\
	&= J^2 n_j |\delta D|\sum_{q\beta k k^\prime \alpha \alpha^\prime,a,b} c^\dagger_{k^\prime\alpha^\prime} c_{k\alpha} S_d^a s^a_{\alpha^\prime \beta}s^b_{\beta \alpha} \frac{\left(\omega - \frac{1}{2}D\right) + J S_d^z \beta}{\left(\omega - \frac{1}{2}D\right)^2 - \frac{1}{16}J^2} S_d^b & \left[s_q^z = \frac{\beta}{2}\right]  \\
	&= \frac{J^2 n_j |\delta D|\left(\omega - \frac{1}{2}D\right)}{\left(\omega - \frac{1}{2}D\right)^2 - \frac{1}{16}J^2} \sum_{k k^\prime \alpha \alpha^\prime,a,b} c^\dagger_{k^\prime\alpha^\prime} c_{k\alpha} S_d^a  S_d^b \left(s^a s^b\right)_{\alpha^\prime \alpha} & \left[\text{term 1}\right]\\
	&+ \frac{J^3 n_j |\delta D|}{\left(\omega - \frac{1}{2}D\right)^2 - \frac{1}{16}J^2} \sum_{\beta k k^\prime \alpha \alpha^\prime,a,b} c^\dagger_{k^\prime\alpha^\prime} c_{k\alpha} S_d^a S_d^z S_d^b s^a_{\alpha^\prime \beta} s^b_{\beta \alpha} \beta & \left[\text{term 2}\right] \\
\end{align}
We will now simplify the terms individually. In term 1, only the configurations \(a \neq b\) can lead to a non-trivial impurity spin operator and hence contribute to renormalization. For \(a \neq b\), we have \(S_d^a S_d^b = \frac{i}{2} \sum_c \epsilon^{abc}S_d^c \). Therefore,
\begin{equation}\begin{aligned}
\sum_{a,b}S_d^a S_d^b\left(s^a s^b\right)_{\alpha^\prime \alpha} = - \frac{1}{4}\sum_{c,e} S_d^c \left(s^e\right)_{\alpha^\prime \alpha}\sum_{a,b}\epsilon^{abc}\epsilon^{abe} = - \frac{1}{4}\sum_{c,e} S_d^c \left(s^e\right)_{\alpha^\prime \alpha} 2 \delta_{ce} = -\frac{1}{2}\vec{S_d}\cdot\vec{s}_{\alpha^\prime \alpha}
\end{aligned}\end{equation}
Term 1 therefore simplifies to
\begin{equation}\begin{aligned}
	-\frac{1}{2}\frac{J^2 n_j \left(\omega - \frac{1}{2}D\right)|\delta D|}{\left(\omega - \frac{1}{2}D\right)^2 - \frac{1}{16}J^2} \sum_{k k^\prime \alpha \alpha^\prime,c} c^\dagger_{k^\prime\alpha^\prime} c_{k\alpha} \vec{S_d}\cdot\vec{s}_{\alpha^\prime \alpha} && \left[\text{term 1}\right]\\
\end{aligned}\end{equation}
In term 2, we use the identity:
\begin{equation}\begin{aligned}
S^a S^z S^b = \frac{1}{4}\sum_c \left(\delta_{ac}\delta_{zb} - \delta_{ab}\delta_{zc}\right) S^c
\end{aligned}\end{equation}
Substituting this gives
\begin{equation}\begin{aligned}
	\sum_{\beta,a,b} S_d^a S_d^z S_d^b s^a_{\alpha^\prime \beta} s^b_{\beta \alpha} \beta &= \frac{1}{4}\sum_{\beta,c}\beta \left(S_d^c s^c_{\alpha^\prime\beta}s^z_{\beta\alpha} - S_d^z s^c_{\alpha^\prime\beta} s^c_{\beta\alpha}\right) \\
	\implies \text{term 2} &= \frac{J^3|\delta D| n_j}{\left(\omega - \frac{1}{2}D\right)^2 - \frac{1}{16}J^2} \frac{1}{4}\sum_{\beta k k^\prime \alpha \alpha^\prime,c} c^\dagger_{k^\prime\alpha^\prime} c_{k\alpha} \beta s^c_{\alpha^\prime\beta}\left(S_d^c s^z_{\beta\alpha} - S_d^z s^c_{\beta\alpha}\right)\\
			       &= \frac{J^3|\delta D| n_j}{\left(\omega - \frac{1}{2}D\right)^2 - \frac{1}{16}J^2} \frac{1}{4}\sum_{\beta k k^\prime \alpha \alpha^\prime,c} c^\dagger_{k^\prime\alpha^\prime} c_{k\alpha} \left(\frac{1}{2}S_d^c s^c_{\alpha^\prime\alpha}\ - \beta S_d^z s^c_{\beta\alpha}s^c_{\alpha^\prime\beta}\right)
\end{aligned}\end{equation}

\subsection{First term}
\begin{align}
	&c^\dagger T \frac{1}{\hat \omega - H_D}T^\dagger c \\
	&= J^2\sum_{q\beta k k^\prime \alpha \alpha^\prime} c^\dagger_{q\beta} c_{k\alpha} \vec{S_d}\cdot\vec{s}_{\beta \alpha} \frac{1}{\omega^\prime - \epsilon_q\tau_{q\beta} - J S_d^z s_q^z}c^\dagger_{k^\prime\alpha^\prime} c_{q\beta} \vec{S_d}\cdot\vec{s}_{\alpha^\prime \beta}\\
	&= J^2\sum_{q\beta k k^\prime \alpha \alpha^\prime a b} c^\dagger_{q\beta} c_{k\alpha} S_d^a s^a_{\beta \alpha} \frac{1}{\omega^\prime - \frac{1}{2}D - J S_d^z s_q^z}c^\dagger_{k^\prime\alpha^\prime} c_{q\beta} S_d^b s^b_{\alpha^\prime \beta} & \left[\tau = -\frac{1}{2},\epsilon_q = -D\right] \\
							   &= J^2\sum_{q\beta k k^\prime \alpha \alpha^\prime a b} c^\dagger_{q\beta} c_{k\alpha} S_d^a s^a_{\beta \alpha} \frac{\left(\omega^\prime - \frac{1}{2}D\right) + J S_d^z s_q^z}{\left(\omega^\prime - \frac{1}{2}D\right)^2 - \frac{1}{16}J^2}c^\dagger_{k^\prime\alpha^\prime} c_{q\beta} S_d^b s^b_{\alpha^\prime \beta}\\
							   &= J^2\sum_{q\beta k k^\prime \alpha \alpha^\prime a b} c^\dagger_{q\beta} c^\dagger_{k^\prime\alpha^\prime}c_{k\alpha} S_d^a s^a_{\beta \alpha} \frac{-\left(\omega^\prime - \frac{1}{2}D\right) - J S_d^z s_q^z}{\left(\omega^\prime - \frac{1}{2}D\right)^2 - \frac{1}{16}J^2} c_{q\beta} S_d^b s^b_{\alpha^\prime \beta} & \left[c_k c^\dagger_{k^\prime} \sim - c^\dagger_{k^\prime}c_k\right] \\
							   &= J^2\sum_{q\beta k k^\prime \alpha \alpha^\prime a b} c^\dagger_{k^\prime\alpha^\prime}c_{k\alpha} S_d^a s^a_{\beta \alpha} \frac{-\left(\omega^\prime - \frac{1}{2}D\right) \hat n_{q\beta} - J S_d^z  c^\dagger_{q\beta}s_q^z c_{q\beta}}{\left(\omega^\prime - \frac{1}{2}D\right)^2 - \frac{1}{16}J^2} S_d^b s^b_{\alpha^\prime \beta} \\
							   &= J^2\sum_{q\beta k k^\prime \alpha \alpha^\prime a b} c^\dagger_{k^\prime\alpha^\prime}c_{k\alpha} S_d^a s^a_{\beta \alpha} \frac{-\left(\omega^\prime - \frac{1}{2}D\right) \hat n_{q\beta} - J S_d^z \left(s_q^z - \frac{\beta}{2}\right) \hat n_{q\beta}}{\left(\omega^\prime - \frac{1}{2}D\right)^2 - \frac{1}{16}J^2} S_d^b s^b_{\alpha^\prime \beta} & \left[\left[c^\dagger_{q\beta}, s_q^z\right] = -\frac{\beta}{2} c^\dagger_{q\beta}\right] \\
							   &= J^2 n_j |\delta D|\sum_{\beta k k^\prime \alpha \alpha^\prime a b} c^\dagger_{k^\prime\alpha^\prime}c_{k\alpha} S_d^a s^a_{\beta \alpha} \frac{-\left(\omega^\prime - \frac{1}{2}D\right) + J S_d^z \beta}{\left(\omega^\prime - \frac{1}{2}D\right)^2 - \frac{1}{16}J^2} S_d^b s^b_{\alpha^\prime \beta} & \left[s_q^z = -\frac{1}{2}\beta\right] \\
							   &= -\frac{J^2\left(\omega^\prime - \frac{1}{2}D\right) n_j |\delta D|}{\left(\omega^\prime - \frac{1}{2}D\right)^2 - \frac{1}{16}J^2}\sum_{k k^\prime \alpha \alpha^\prime a b} c^\dagger_{k^\prime\alpha^\prime}c_{k\alpha} S_d^a  S_d^b \left(s^b s^a\right)_{\alpha^\prime \alpha} & \left[\text{term 3}\right] \\
							   &\quad+ \frac{J^3 n_j |\delta D|}{\left(\omega^\prime - \frac{1}{2}D\right)^2 - \frac{1}{16}J^2} \sum_{q\beta k k^\prime \alpha \alpha^\prime a b} c^\dagger_{k^\prime\alpha^\prime}c_{k\alpha} S_d^a S_d^z S_d^b s^a_{\beta \alpha} s^b_{\alpha^\prime \beta} \beta & \left[\text{term 4}\right]
\end{align}
term 3 can be made identical to term 1 using the relation: \(s^b s^a = -s^b s^a \text{ for } a \neq b\). With this change, term 3 becomes
\begin{equation}\begin{aligned}
	\text{term 3} = \frac{J^2 n_j \left(\omega^\prime - \frac{1}{2}D\right)|\delta D|}{\left(\omega^\prime - \frac{1}{2}D\right)^2 - \frac{1}{16}J^2}\sum_{k k^\prime \alpha \alpha^\prime a b} c^\dagger_{k^\prime\alpha^\prime}c_{k\alpha} S_d^a  S_d^b \left(s^a s^b\right)_{\alpha^\prime \alpha} \\
	= \frac{1}{2}\frac{J^2 n_j \left(\omega^\prime - \frac{1}{2}D\right)|\delta D|}{\left(\omega^\prime - \frac{1}{2}D\right)^2 - \frac{1}{16}J^2} \sum_{k k^\prime \alpha \alpha^\prime,c} c^\dagger_{k^\prime\alpha^\prime} c_{k\alpha} \vec{S_d}\cdot\vec{s}_{\alpha^\prime \alpha}
\end{aligned}\end{equation}

For term 4, we get
\begin{equation}\begin{aligned}
	\sum_{\beta,a,b} S_d^a S_d^z S_d^b s^b_{\alpha^\prime \beta} s^a_{\beta \alpha} \beta = \frac{1}{4}\sum_{\beta,a,b,c}\beta\left(\delta_{ac}\delta_{zb} - \delta_{ab}\delta_{zc}\right) S^cs^b_{\alpha^\prime \beta} s^a_{\beta \alpha} = \frac{1}{4}\sum_{\beta,c}\beta s^c_{\beta\alpha}\left(S^c_d s^z_{\alpha^\prime\beta} - S^z_d s^c_{\alpha^\prime\beta}\right)
\end{aligned}\end{equation}
which gives
\begin{equation}\begin{aligned}
	\text{term 4} = \frac{J^3 n_j |\delta D|}{\left(\omega^\prime - \frac{1}{2}D\right)^2 - \frac{1}{16}J^2} \frac{1}{4}\sum_{\beta k k^\prime \alpha \alpha^\prime,c} c^\dagger_{k^\prime\alpha^\prime} c_{k\alpha} \left(\frac{1}{2}S_d^c s^c_{\alpha^\prime\alpha}\ - \beta S_d^z s^c_{\beta\alpha}s^c_{\alpha^\prime\beta}\right)
\end{aligned}\end{equation}

\subsection{Total renormalization \(\Delta H^{(2)}\)}
From the formula for the renormalization \(\Delta H^{(2)}\), we write
\begin{equation}\begin{aligned}
	\Delta H^{(2)} = \text{term 3} + \text{term 4} - \text{term 1} - \text{term 2}
\end{aligned}\end{equation}
From the constraints of URG and particle-hole symmetry, we have the constraint \(\omega + \omega^\prime = H_d^0 + H_d^1\). \(H_d^0\) is the diagonal part when the current node is unoccupied and \(H_d^1\) is when its occupied. For our case, \(H_d^0 + H_d^1 = \sum \epsilon_q \tau_q = D\), because both the hole and particle states have energy of \(\frac{D}{2}\) (\(\epsilon_q\) and \(\tau\) flip sign together). We therefore have \((\omega - D/2) = -(\omega^\prime - D/2)\). term 2 and term 4 cancel each other and term 3 becomes equal to term 1. The total renormalization at second order is therefore (relabelling \(\omega^\prime\) as \(\omega\))
\begin{equation}\begin{aligned}
	\Delta H^{(2)} = \text{term 3} = \frac{1}{2}\frac{J^2 n_j \left(\omega - \frac{1}{2}D\right)|\delta D|}{\left(\omega - \frac{1}{2}D\right)^2 - \frac{1}{16}J^2} \sum_{k k^\prime \alpha \alpha^\prime,c} c^\dagger_{k^\prime\alpha^\prime} c_{k\alpha} \vec{S_d}\cdot\vec{s}_{\alpha^\prime \alpha}
\end{aligned}\end{equation}
which gives
\begin{equation}\begin{aligned}
	\Delta J^{(2)} = \frac{1}{2}\frac{J^2 n_j \left(\omega - \frac{1}{2}D\right)|\delta D|}{\left(\omega - \frac{1}{2}D\right)^2 - \frac{1}{16}J^2}
\end{aligned}\end{equation}
The choice of \(\omega = D\) gives
\begin{equation}\begin{aligned}
	\Delta J^{(2)} = \frac{J^2 D  n_j |\delta D|}{D^2 - \frac{1}{4}J^2}
\end{aligned}\end{equation}
For \(J \ll D\), we get the one-loop PMS form.

\end{document}
